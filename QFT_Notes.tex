\documentclass[11pt]{article}
\usepackage[letterpaper,margin=0.75in,left=1.25in,nohead]{geometry}
% \usepackage{graphicx}
% \usepackage{color}
\usepackage{amsmath} % align environment
% \usepackage{amsfonts}
\usepackage{amssymb}
% \usepackage{bbm}
% \usepackage{tikz}
% \usetikzlibrary{intersections}
% \usepackage{subcaption}
% \usepackage{mathrsfs}
\usepackage{braket}

% Command definitions for QFT Notes

\newcommand{\ScalarField}{\phi}
\newcommand{\FlatMetric}{\eta}
\newcommand{\Action}{S}
\newcommand{\LagrangianDensity}{\mathcal{L}}
\newcommand{\DD}[1]{\partial_{#1}} % Directional derivative
\newcommand{\Dal}{\square} % d'Alembertian (Box) operator
\newcommand{\TimeEvol}{U}
\newcommand{\Hamiltonian}{H}
\newcommand{\Prop}{K}

\begin{document}

\title{QFT Notes}
\author{Kevin Roebuck}
\maketitle

%

\section{Class 1 - 3/19}
\subsection{Notation and general setup}
We'll work with a neutral scalar field, $\ScalarField$, that is self-interacting. On top of this, we will stay in flat space and use the signature $\FlatMetric_{\mu\nu} = diag(-1, 1, 1, 1)$. Generic coordinates will be denoted by $x$ such that $x^\mu = (t, x, y, z)$ for $\mu = 0, 1, 2, 3$. The distinction between the generic coordinate $x$ and $x^1 = x$ will generally be reliant on the context of the situation at hand. Regarding units, $c=1$ will always be the case, while $\hbar=1$ will only be the case most of the time.

The action is given by
\begin{align}
\Action = \int d^4 x \left[\frac{1}{2}\FlatMetric^{\alpha\beta}(\DD{\alpha}\ScalarField)(\DD{\beta}\ScalarField) + \frac{1}{2}m^2\ScalarField^2 + V(\ScalarField)\right].
\end{align}
Typically, the potential will look like
\begin{align}
V(\ScalarField) = \frac{\lambda}{4}\ScalarField^4.
\end{align}
When using the variational principle, we will assume that
\begin{align}
\frac{\delta\ScalarField(x)}{\delta\ScalarField(x')} = \delta(x-x') = \delta(t-t')\delta(\vec{x}-\vec{x}').
\end{align}
If we set the variation of the action with respect to our scalar field equal to zero, we get a wave equation
\begin{align}
\frac{\delta\Action}{\delta\ScalarField} = 0 \qquad\rightarrow\qquad \Dal\ScalarField - m^2\ScalarField - V'(\ScalarField) = 0,
\end{align}
where $\Dal \equiv \DD{t}^2 + \nabla^2$.

\subsection{Vacuum persistence amplitude}
In QFT, we will be looking at the vacuum persistence amplitude
\begin{align}
\braket{0_{-}|0_{+}},
\end{align}
which is related to the probability that the vacuum stays a vacuum. Here, $0_{-}$ ($0_{+}$) is the in (out) vacuum state. The amplitude is given by
\begin{align}
\braket{0_{-}|0_{+}} = \int [d\ScalarField] e^{i\Action[\ScalarField]/\hbar}.
\end{align}
Heuristically, this means that we should take all possible real functions, $\ScalarField$, evaluate the action for each function, and then add them all up. How can we do this? The answer is that we will discretize the spacetime. For simplicity, let us imagine we are only in 2 dimensions $(t, x)$, so that at $(t_i, x_j)$, $\ScalarField$ is given by $\ScalarField_{i,j}$. The action can then be written as
\begin{align}
\Action[\ScalarField] = \sum_i\sum_j \left[
\frac{1}{2}\left(\frac{\ScalarField_{i,j} - \ScalarField_{i-1,j}}{\Delta t}\right)^2
- \frac{1}{2}\left(\frac{\ScalarField_{i,j} - \ScalarField_{i,j-1})}{\Delta x}\right)^2
+ m^2\ScalarField^2_{i,j} + V(\ScalarField_{i,j})\right]\Delta x \Delta t
\end{align}


\end{document}